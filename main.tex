%%%%%%%%%%%%%%%%%
% This is an sample CV template created using altacv.cls
% (v1.1.5, 1 December 2018) written by LianTze Lim (liantze@gmail.com). Now compiles with pdfLaTeX, XeLaTeX and LuaLaTeX.
%
%% It may be distributed and/or modified under the
%% conditions of the LaTeX Project Public License, either version 1.3
%% of this license or (at your option) any later version.
%% The latest version of this license is in
%%    http://www.latex-project.org/lppl.txt
%% and version 1.3 or later is part of all distributions of LaTeX
%% version 2003/12/01 or later.
%%%%%%%%%%%%%%%%

%% If you need to pass whatever options to xcolor
\PassOptionsToPackage{dvipsnames}{xcolor}

%% If you are using \orcid or academicons
%% icons, make sure you have the academicons
%% option here, and compile with XeLaTeX
%% or LuaLaTeX.
% \documentclass[10pt,a4paper,academicons]{altacv}

%% Use the "normalphoto" option if you want a normal photo instead of cropped to a circle
% \documentclass[10pt,a4paper,normalphoto]{altacv}

\documentclass[11pt,b4paper,ragged2e]{altacv}


%% AltaCV uses the fontawesome and academicon fonts
%% and packages.
%% See texdoc.net/pkg/fontawecome and http://texdoc.net/pkg/academicons for full list of symbols. You MUST compile with XeLaTeX or LuaLaTeX if you want to use academicons.

% Change the page layout if you need to
\geometry{left=1cm,right=9cm,marginparwidth=6.8cm,marginparsep=1.2cm,top=1.25cm,bottom=1.25cm}

% Change the font if you want to, depending on whether
% you're using pdflatex or xelatex/lualatex
\ifxetexorluatex
  % If using xelatex or lualatex:
  \setmainfont{Carlito}
\else
  % If using pdflatex:
  \usepackage[utf8]{inputenc}
  \usepackage[T1]{fontenc}
  \usepackage[default]{lato}
  \usepackage{hyperref}
\fi

% Change the colours if you want to
\definecolor{Mulberry}{HTML}{72243D}
\definecolor{SlateGrey}{HTML}{2E2E2E}
\definecolor{LightGrey}{HTML}{666666}
\colorlet{heading}{Blue}
\colorlet{accent}{Blue}
\colorlet{emphasis}{SlateGrey}
\colorlet{body}{LightGrey}

% Change the bullets for itemize and rating marker
% for \cvskill if you want to
\renewcommand{\itemmarker}{{\small\textbullet}}
\renewcommand{\ratingmarker}{\faCircle}

%% sample.bib contains your publications
\addbibresource{sample.bib}

\begin{document}
\name{Vishu Tyagi}
\tagline{B. Tech. IT Graduate with interest in Machine Learning and Backend Development}
%\photo{2.8cm}{Globe_High}
\personalinfo{%
  % Not all of these are required!
  % You can add your own with \printinfo{symbol}{detail}
  \email{vishutyagi.018@gmail.com}
  \phone{(+91) 860-728-7966}
%   \mailaddress{Sonipat, Harayana}
  \location{Sonipat, Haryana}
  %\homepage{www.homepage.com/}
  %\twitter{@twitterhandle}
  \linkedin{linkedin.com/in/strider187}
  \github{github.com/strider187}
  %% You MUST add the academicons option to \documentclass, then compile with LuaLaTeX or XeLaTeX, if you want to use \orcid or other academicons commands.
  % \orcid{orcid.org/0000-0000-0000-0000}
}

%% Make the header extend all the way to the right, if you want.
\begin{fullwidth}
\makecvheader
\end{fullwidth}

%% Depending on your tastes, you may want to make fonts of itemize environments slightly smaller
% \AtBeginEnvironment{itemize}{\small}

%% Provide the file name containing the sidebar contents as an optional parameter to \cvsection.
%% You can always just use \marginpar{...} if you do
%% not need to align the top of the contents to any
%% \cvsection title in the "main" bar.

%%%%%%%%%%%%%%%%%%%%%Experience%%%%%%%%%%%%%%%%%%%%%%%%%%%%%%%%%%%%%
\cvsection[page1sidebar]{Experience}
\cvevent{Data Engineer}{Wizely}{Present}{Bengaluru}
\begin{itemize}
    \item Working on the development of the new data pipeline based on Python, Mongo DB, PostgreSQL and Big Query to properly manage the NeoBank data.
    \item Working on development of the various reporting layers to help the Business and Analytic teams better analyze the impact of the product better.
    \end{itemize} \medskip
    

\cvevent{Co-Op}{Info Edge India Ltd.}{December \textnormal{2020} -  \textnormal{June 2021}}{Noida}
\begin{itemize}
    \item Worked as a part of the Data Science Team in Info Edge which developed the new CV Parser Version 4 for the Naukri department.
    \item Worked on EDA and ETL using methods available in Pandas and visualization techniques available in Seaborn like Scatter Plot, HeatMap, and Barplot.
    \item Used NLP and Neural Networks for developing complex Deep Learning Models used for the NER.
    \end{itemize} \medskip
    
    
\cvevent{Software Engineer}{Criddle}{October \textnormal{2020} -  \textnormal{
December \textnormal{2020}}}{Bengaluru}
\begin{itemize}
    \item Worked on the Backend Development of the social movie review Application using Django Framework with PostgreSQL.
    \item Developed a number of APIs for various new features like Reviews and Feed Model in the App. Also redesigned the recommendation building APIs.
    \item Worked on shifting the App from DigitalOcean to Amazon AWS using an EC2 AMI and a RDS Postgres Database.
    \end{itemize} \medskip
    
\cvevent{Software Development Intern}{Hindustan Unilever Ltd.}{June \textnormal{ 2019} - \textnormal{August} \textnormal{ 2019}}{Bengaluru}
\begin{itemize}
    \item Developed software to solve the last mile delivery issue to keep track of the number of man-hours spent by workers in their designated space to increase efficiency.
    \item Used advanced models based on Machine Learning and Image Processing tools like Yolov3 and Centroid Tracker with Django Backend.
    \item The software was deployed on a dedicated RS point and was able to automate the complete process for the analysis.
    \end{itemize} \medskip
    
\cvevent{Machine Learning Intern}{Suvidha (Startup under Cocoberry Restaurants & Distributions PVT Ltd.)}{June \textnormal{ 2018} - \textnormal{July} \textnormal{ 2018}}{New Delhi}
\begin{itemize}
    \item Completed work as a Machine Learning Intern by developing a Face Detection Model.
    \item The model is based upon Faceboxes: A CPU Real-time Face Detector with High Accuracy.
\end{itemize}




%%%%%%%%%%%%%Education%%%%%%%%%%%%%%%%%%%%%%%%%%%%%%%%%%%%%%%%%%%%%







%%%%%%%%%%%%%%%%Projects%%%%%%%%%%%%%%%%%%%%%%%%%%%%%%%%%5
\cvsection{Projects}
% \cvevent{Image Segmentation Using ESPNET}{USIC\&T}{October 2018}{}
% \begin{itemize}
% \item \small Implemented the newly developed fast and efficient convolutional neural network, ESPNet, for semantic segmentation of high-resolution images.
% \item \small The link to the paper which inspired the project is: \href{https://arxiv.org/abs/1803.06815}{ESPNet: Efficient Spatial Pyramid of Dilated Convolutions for Semantic Segmentation}
% \end{itemize} \vspace{-3 mm}
% \divider\newline\vspace{-1 mm}
% \cvevent{Transfer Learning for NLP with Tensorflow}{Coursera and Kaggle}{November 2020}{} 
% \begin{itemize}
%     \item Developed a NLP model for the \href{https://www.kaggle.com/c/quora-insincere-questions-classification/data}{Quora Insincere Questions Classification Problem} on Kaggle.
%     \item Used Embeddings from Language Models, Universal Sentence Encoders and NNLM layers TensorHub to develop a model.
%     \item Achieved a maximum accuracy of 96.17 on Training Data and 95.2 on Validation data with Universal Sentence Encoder.
% \end{itemize}
% \end{itemize}

\cvevent{\textcolor{Blue}{Wordonary}}{}{May 2021}{} 
\begin{itemize}
    \item Built the backend of a specialized Dictionary App offering a number of new features and developed a number of APIs to search, save and extract meanings. \href{http://wordonary.co.in/}{\textcolor{blue}{Try here.}}
    \item Used a custom MySQL database in place of the default sqlite database used in Django.
    \item Deployed the App's backend on AWS using an EC2 instance and a MySQL RDS database.
    \item Developed a scraper in order to search for words which were not currently present in the database and add them.
\end{itemize}





%%%%%%%%%%%%%%%%%%%%%%%%%%%%%%%%%%%%%%%%%%%%%%%%%%%%%%%%%%%%%%%%%%%%%%

%%%%%%%%%%%%%%%%%Day of life%%%%%%%%%%%%%%%%%%%%
%\cvsection{A Day of My Life}

% Adapted from @Jake's answer from http://tex.stackexchange.com/a/82729/226
% \wheelchart{outer radius}{inner radius}{
% comma-separated list of value/text width/color/detail}
%\wheelchart{1.5cm}{0.5cm}{%
 % 6/8em/accent!30/{Sleep,\\beautiful sleep},
  %3/8em/accent!40/Hopeful novelist by night,
  %8/8em/accent!60/Daytime job,
  %2/10em/accent/Sports and relaxation,
  %5/6em/accent!20/Spending time with family
% }

%\cvsection[page2sidebar]{Publications}

%\nocite{*}

%\printbibliography[heading=pubtype,title={\printinfo{\faBook}{Books}},type=book]

%\divider

%\printbibliography[heading=pubtype,title={\printinfo{\faFileTextO}{Journal %Articles}},type=article]

%\divider

%\printbibliography[heading=pubtype,title={\printinfo{\faGroup}{Conference %Proceedings}},type=inproceedings]

%% If the NEXT page doesn't start with a \cvsection but you'd
%% still like to add a sidebar, then use this command on THIS
%% page to add it. The optional argument lets you pull up the
%% sidebar a bit so that it looks aligned with the top of the
%% main column.
% \addnextpagesidebar[-1ex]{page3sidebar}

\footnotetext[1]{USIC\&T, GGSIPU follows absolute point grading system.}
%\footnotetext[2]{CGPA available for the first 5 semesters, result awaited for 6^{th}}
\clearpage
\end{document}
